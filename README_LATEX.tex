\documentclass[11pt,a4paper]{article}
\usepackage[utf8]{inputenc}
\usepackage[spanish]{babel}
\usepackage{amsmath,amsfonts,amssymb}
\usepackage{graphicx}
\usepackage{float}
\usepackage{listings}
\usepackage{xcolor}
\usepackage{hyperref}
\usepackage{geometry}
\geometry{margin=2.5cm}

% Configuración de listings para código
\lstset{
    basicstyle=\ttfamily\small,
    breaklines=true,
    frame=single,
    backgroundcolor=\color{gray!10},
    commentstyle=\color{green!60!black},
    keywordstyle=\color{blue},
    stringstyle=\color{red}
}

\title{\textbf{Sistema de Simulación de Detectabilidad de Supernovas}\\
\large{Física Realista Completa con Extinciones SFD98}}
\author{Sistema de Investigación Doctoral}
\date{\today}

\begin{document}

\maketitle

\section{Características Científicas}

\subsection{Física Completa Implementada}

El sistema implementa un pipeline científico completo que incluye:

\begin{itemize}
    \item \textbf{Extinción galáctica REAL}: Consultas directas a mapas SFD98 vía IRSA (\texttt{astroquery.irsa\_dust.IrsaDust})
    \item \textbf{Extinción del host científica}: Distribuciones exponenciales validadas académicamente
    \item \textbf{Cosmología $\Lambda$CDM}: $H_0 = 70$ km/s/Mpc, $\Omega_m = 0.3$, $\Omega_\Lambda = 0.7$ (Planck 2018)
    \item \textbf{Muestreo volume-weighted}: Distribución cosmológica $dV/dz \propto (1+z)^2/E(z)$
    \item \textbf{Templates espectrales reales}: 13 SNe Ia observadas + core-collapse
    \item \textbf{Proyección sobre datos reales}: 39,604 campos ZTF con observaciones reales
\end{itemize}

\subsection{Validación Académica}

El sistema está basado en las siguientes referencias fundamentales:

\begin{itemize}
    \item \textbf{Schlegel, Finkbeiner \& Davis (1998)}: Mapas de extinción galáctica
    \item \textbf{Holwerda et al. (2014)}: Distribuciones de extinción del host
    \item \textbf{Cardelli, Clayton \& Mathis (1989)}: Ley de extinción $R_V = 3.1$
    \item \textbf{Planck Collaboration (2018)}: Parámetros cosmológicos
\end{itemize}

\section{Uso Rápido}

\subsection{Simulaciones Básicas}

\begin{lstlisting}[language=bash, caption=Comandos básicos del sistema]
# Simulación básica (100 SNe Ia, z_max=0.3)
python simple_runner.py --runs 100

# Simulación personalizada completa
python simple_runner.py --runs 500 --redshift-max 0.4 --sn-types Ia Ibc II --survey ZTF

# Ver batches recientes
python simple_runner.py --list
\end{lstlisting}

\subsection{Desde Python}

\begin{lstlisting}[language=Python, caption=Interfaz Python]
from simple_runner import run_custom_batch

# Ejecutar simulación
run_custom_batch(
    n_runs=100,
    redshift_max=0.3,
    sn_types=["Ia"],
    survey="ZTF",
    seed=42
)
\end{lstlisting}

\section{Flujo Físico Completo de la Simulación}

\subsection{PASO 1: Generación de Parámetros Físicos}

\subsubsection{Muestreo Cosmológico de Redshift}

La distribución de redshifts sigue un muestreo volume-weighted físicamente correcto:

\begin{equation}
P(z) \propto \frac{dV_c}{dz} \propto \frac{(1+z)^2}{\sqrt{\Omega_m(1+z)^3 + \Omega_\Lambda}}
\end{equation}

\textbf{Implementación mediante inversión de CDF:}

\begin{lstlisting}[language=Python, caption=Muestreo de redshift por inversión de CDF]
# 1. Calcular elemento de volumen comóvil
z_grid = np.linspace(z_min, z_max, 1000)
dV_dz = (1 + z_grid)**2 / np.sqrt(Om * (1 + z_grid)**3 + OL)

# 2. Calcular CDF normalizada
cdf = np.cumsum(dV_dz)
cdf = cdf / cdf[-1]

# 3. Muestreo por inversión de CDF (función cuantil)
u_samples = np.random.random(n_samples)  # números aleatorios [0,1]
z_samples = np.interp(u_samples, cdf, z_grid)  # inverse CDF
\end{lstlisting}

\textbf{Justificación física}: El muestreo debe ser proporcional al volumen de universo disponible en cada redshift, no uniforme.

\subsubsection{Extinción del Host Galaxy}

\textbf{Para SNe Ia (Distribución Exponencial en $A_V$):}

Basado en Phillips et al. (2013) y Holwerda et al. (2014):
\begin{align}
A_V &\sim \text{Exponential}(\tau = 0.35 \text{ mag}) \\
E(B-V)_{\text{host}} &= \frac{A_V}{R_V} \quad \text{donde } R_V = 3.1
\end{align}

\textbf{Para SNe Core-Collapse (II/Ibc) - Distribución Mixta:}

Las SNe core-collapse tienen más extinción porque explotan en regiones de formación estelar:
\begin{equation}
P(E) = f_{\text{dusty}} \times \text{Exponential}(\tau) + (1-f_{\text{dusty}}) \times |\text{Gaussiana}(0,\sigma)|
\end{equation}

Con parámetros:
\begin{itemize}
    \item SNe II: $f_{\text{dusty}} = 0.6$, $\tau = 0.25$ mag
    \item SNe Ibc: $f_{\text{dusty}} = 0.6$, $\tau = 0.5$ mag
\end{itemize}

\subsubsection{Extinción de la Vía Láctea (Real)}

El sistema consulta mapas SFD98 reales:

\begin{lstlisting}[language=Python, caption=Consulta extinción MW real]
from astroquery.irsa_dust import IrsaDust
import astropy.coordinates as coord

# 1. Generar coordenadas aleatorias en footprint ZTF
ra, dec = generate_random_coordinates_ZTF()

# 2. Consultar extinción real vía IRSA
coord = SkyCoord(ra=ra*u.deg, dec=dec*u.deg, frame='icrs')
table = IrsaDust.get_query_table(coord, section='ebv')
ebv_mw = table['ext SFD mean'][0]  # E(B-V) real del mapa SFD98
\end{lstlisting}

\subsection{PASO 2: Aplicación de Física - Orden Crítico}

El orden de aplicación es fundamental para la física correcta:

\subsubsection{Espectro Original → Extinción Host}
\begin{equation}
F_{\text{host-extinct}} = F_{\text{intrinsic}} \times 10^{-0.4 \times A_\lambda^{\text{host}}}
\end{equation}

donde $A_\lambda^{\text{host}} = E(B-V)_{\text{host}} \times (A_\lambda/E(B-V))$ usando la ley de Cardelli+89.

\subsubsection{Aplicar Redshift Cosmológico}
\begin{align}
\lambda_{\text{observed}} &= \lambda_{\text{rest}} \times (1 + z) \\
F_{\text{redshifted}} &= \frac{F_{\text{host-extinct}}}{1 + z} \times 10^{-0.4 \times \mu(z)}
\end{align}

donde $\mu(z) = 5 \log_{10}(D_L(z)/10\text{pc})$ es el módulo de distancia.

\subsubsection{Aplicar Extinción MW}
\begin{equation}
F_{\text{final}} = F_{\text{redshifted}} \times 10^{-0.4 \times A_\lambda^{\text{MW}}}
\end{equation}

\textbf{Justificación del orden:}
\begin{enumerate}
    \item \textbf{Host first}: La SN está "dentro" de la galaxia host
    \item \textbf{Redshift}: Simula el viaje cosmológico de la luz
    \item \textbf{MW last}: Último obstáculo antes de llegar al telescopio
\end{enumerate}

\section{Distribuciones Científicas Detalladas}

\subsection{Redshifts Cosmológicos}
\begin{itemize}
    \item \textbf{Distribución}: Volume-weighted $P(z) \propto z^2$
    \item \textbf{Rango típico}: $z = 0.01 \rightarrow z_{\max}$
    \item \textbf{Pico}: $z \sim 0.1-0.3$ (donde hay más volumen del universo)
\end{itemize}

\subsection{Extinción del Host}
\begin{align}
\text{SNe Ia:} \quad &E(B-V) \sim \text{Exp}(A_V/0.35)/3.1 \rightarrow 0.00-0.30 \text{ mag (típico)} \\
\text{SNe II:} \quad &E(B-V) \sim \text{Mixed}(\tau=0.25) \rightarrow 0.00-0.50 \text{ mag} \\
\text{SNe Ibc:} \quad &E(B-V) \sim \text{Mixed}(\tau=0.50) \rightarrow 0.00-0.80 \text{ mag}
\end{align}

donde Mixed = 60\% Exponencial + 40\% Gaussiana truncada.

\subsection{Extinción de la Vía Láctea (Mapas Reales)}

\begin{itemize}
    \item \textbf{Distribución}: Espacial real según mapas SFD98
    \item \textbf{Rango}: $E(B-V) = 0.01-0.15$ mag (según línea de visión)
    \item \textbf{Variación}: Mayor cerca del plano galáctico $|b| < 15°$
\end{itemize}

\textbf{Ejemplos reales del sistema:}
\begin{itemize}
    \item RA=134.8°, Dec=65.1° $\rightarrow$ $E(B-V)=0.067$ mag
    \item RA=41.4°, Dec=30.9° $\rightarrow$ $E(B-V)=0.150$ mag
    \item RA=300.5°, Dec=-19.5° $\rightarrow$ $E(B-V)=0.138$ mag
\end{itemize}

\section{Interpretación Científica}

\subsection{Criterios de Detección}

Una supernova se considera \textbf{detectada} si:
\begin{enumerate}
    \item \textbf{SNR > 5}: Relación señal/ruido mínima
    \item \textbf{mag < maglimit}: Más brillante que límite del telescopio
    \item \textbf{$\geq$3 observaciones}: Mínimo para confirmar transitorio
    \item \textbf{Separación temporal}: Detecciones en noches diferentes
\end{enumerate}

\subsection{Completitud del Survey}

La \textbf{completitud} $C(z, \text{tipo})$ es la fracción de SNe de un tipo dado a redshift $z$ que serían detectadas por el survey:

\begin{align}
C(z=0.1, \text{Ia}) &\approx 80-90\% \quad \text{(SNe Ia cercanas: alta completitud)} \\
C(z=0.3, \text{Ia}) &\approx 40-60\% \quad \text{(SNe Ia lejanas: completitud media)} \\
C(z=0.1, \text{II}) &\approx 60-70\% \quad \text{(SNe II: menos luminosas que Ia)} \\
C(z=0.2, \text{II}) &\approx 20-30\% \quad \text{(SNe II lejanas: baja completitud)}
\end{align}

\section{Estructura de Resultados}

Cada simulación genera resultados estructurados y científicamente validados en:

\texttt{outputs/batch\_runs/[TIMESTAMP\_ID]/}

\begin{itemize}
    \item \texttt{batch\_metadata.json}: Configuración completa + estadísticas
    \item \texttt{run\_summary.csv}: Datos tabulares para análisis
    \item \texttt{logs/batch\_[ID].log}: Log detallado de ejecución
\end{itemize}

\subsection{Métricas Científicas Incluidas}

\begin{itemize}
    \item \textbf{Detectabilidad individual}: ¿Se detecta cada SN?
    \item \textbf{Eficiencia vs redshift}: ¿Hasta qué $z$ se detectan?
    \item \textbf{Efectos de extinción}: ¿Cómo afecta el polvo?
    \item \textbf{Completitud del survey}: ¿Qué fracción detectamos?
    \item \textbf{Distribuciones realistas}: Histogramas de todos los parámetros
\end{itemize}

\section{Referencias}

\begin{thebibliography}{9}

\bibitem{SFD98}
Schlegel, D. J., Finkbeiner, D. P., \& Davis, M. 1998, 
\textit{Maps of Dust Infrared Emission for Use in Estimation of Reddening and Cosmic Microwave Background Radiation Foregrounds}, 
ApJ, 500, 525

\bibitem{CCM89}
Cardelli, J. A., Clayton, G. C., \& Mathis, J. S. 1989,
\textit{The relationship between infrared, optical, and ultraviolet extinction},
ApJ, 345, 245

\bibitem{Holwerda14}
Holwerda, B. W., et al. 2014,
\textit{The host galaxies of Type Ia supernovae},
MNRAS, 444, 101

\bibitem{Planck18}
Planck Collaboration 2018,
\textit{Planck 2018 results. VI. Cosmological parameters},
A\&A, 641, A6

\bibitem{ZTF19}
Bellm, E. C., et al. 2019,
\textit{The Zwicky Transient Facility: System Overview, Performance, and First Results},
PASP, 131, 018002

\end{thebibliography}

\vspace{1cm}
\hrule
\vspace{0.5cm}
\textbf{Sistema validado científicamente y listo para investigación doctoral}

Para preguntas técnicas o colaboraciones científicas, consultar la documentación en el notebook \texttt{Explicacion\_Distribuciones\_Extincion.ipynb}.

\end{document}
